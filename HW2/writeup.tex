\documentclass{article}
\usepackage[english]{babel}
\usepackage{amsmath}
\usepackage{amssymb}
\usepackage{amsthm}
\usepackage{amsfonts}
\usepackage{blindtext}
\usepackage{systeme}
\usepackage{relsize}
\usepackage{enumerate}
\usepackage{commath}
\usepackage{mathrsfs}
\usepackage{bm}
\usepackage{float}
\usepackage{graphicx}
\usepackage{wrapfig}
\usepackage{mathtools}
\usepackage{subcaption}
\usepackage{listings}
\usepackage{color}
\usepackage{tikz-cd}
\usetikzlibrary{automata,positioning}
%\usepackage{hyperref}
%\hypersetup{
% colorlinks=false,% hyperlinks will be black
%linkbordercolor=black,% hyperlink borders will be red
%pdfborderstyle={/S/U/W 0.25}% border style will be underline of width 1pt
%}
\usepackage{chngcntr}
\usepackage{MnSymbol}
\usepackage{tikz}
\usepackage[margin=0.75in]{geometry}
\usepackage{cleveref}

\definecolor{mygreen}{rgb}{0,0.6,0}
\definecolor{mygray}{rgb}{0.5,0.5,0.5}
\definecolor{mymauve}{rgb}{0.58,0,0.82}

\lstset{ %
	backgroundcolor=\color{white},   % choose the background color; you must add \usepackage{color} or \usepackage{xcolor}; should come as last argument
	basicstyle=\ttfamily\footnotesize,        % the size of the fonts that are used for the code
	breakatwhitespace=false,         % sets if automatic breaks should only happen at whitespace
	breaklines=true,                 % sets automatic line breaking
	captionpos=b,                    % sets the caption-position to bottom
	commentstyle=\color{mygreen},    % comment style
	deletekeywords={...},            % if you want to delete keywords from the given language
	escapeinside={\%*}{*)},          % if you want to add LaTeX within your code
	extendedchars=true,              % lets you use non-ASCII characters; for 8-bits encodings only, does not work with UTF-8
	frame=single,	                   % adds a frame around the code
	keepspaces=true,                 % keeps spaces in text, useful for keeping indentation of code (possibly needs columns=flexible)
	keywordstyle=\color{blue},       % keyword style
	language=Octave,                 % the language of the code
	morekeywords={*,...},            % if you want to add more keywords to the set
	numbers=left,                    % where to put the line-numbers; possible values are (none, left, right)
	numbersep=5pt,                   % how far the line-numbers are from the code
	numberstyle=\tiny\color{mygray}, % the style that is used for the line-numbers
	rulecolor=\color{black},         % if not set, the frame-color may be changed on line-breaks within not-black text (e.g. comments (green here))
	showspaces=false,                % show spaces everywhere adding particular underscores; it overrides 'showstringspaces'
	showstringspaces=false,          % underline spaces within strings only
	showtabs=false,                  % show tabs within strings adding particular underscores
	stepnumber=2,                    % the step between two line-numbers. If it's 1, each line will be numbered
	stringstyle=\color{mymauve},     % string literal style
	tabsize=2,	                   % sets default tabsize to 2 spaces
	%title=\lstname                   % show the filename of files included with \lstinputlisting; also try caption instead of title
}

\title{CSCI 3202 Artificial Intelligence Homework 2}
\author{Luke Meszar}
\date{October 27, 2017}
\setcounter{section}{2}
\begin{document}
	\maketitle
	\subsection{(35 Points) Sim Game}
	\lstinputlisting[language=Python]{simWriteup.py}
	The above code is only the important parts of the code I wrote so it will not run as is. The main approach I took in this problem was using alpha-beta pruning to find the best move for the computer. The heuristic I used was the length of legal moves remaining after a  certain number of moves between the computer and player. I choose this heuristic since having the most possible valid moves means that the computer can survive longer. Another possible heuristic is choosing the move that minimizes the number of moves the opponent has. To make this search faster, I started with a small search depth at the beginning of the game and then increased the depth as the game went on and the number of possible moves went down. Also, to improve the efficiency, I didn't pre-generate a game tree. For every possible move at a given level, I simulated playing that move and then continued recursing with alpha-beta. This, effectively allows a DFS approach which is more efficient since branches that can be pruned never have to be fully computed. 
	
	\textbf{Sample Game}
	
	Enter 0 to play as Red, 1 to play as Blue: 0
	
	Red Move: H,D
	
	Blue Move: E,B
	
	Red Move: D,B 
	
	Blue Move: F,B
	
	Red Move: E,F
	
	Blue Move: D,E
	
	Red Move: F,D
	
	Blue Move: G,A 
	
	Red Move: F,G
	
	Blue Move: C,H
	
	Red Move: F,C
	
	Blue Move: E,A
	
	Red Move: G,H
	
	Blue Move: C,B
	
	Red Move: H,E
	
	Blue Move: F,E
	
	Red Move: B,G
	
	Blue Move: G,C
	
	Red Move: F,A
	
	Blue Move: D,C
	
	Red Move: B,A
	
	Blue Move:  A,H
	
	No valid moves for Red. 
	
	Game over. Blue Wins!
	\subsection{(15 points) Alpha-Beta Pruning}
	\begin{enumerate}[(a)]
		\item The overall value to MAX is 3. 
		\item The number of nodes that still need to be evaluated is 49.
		\item The number of nodes that still need to be evaluated is 53.
		\item The number of nodes that still need to be evaluated is 49.
	\end{enumerate}
	\setcounter{section}{2}
	\section{(20 points) 4x4 Sudoku}
	\begin{enumerate}
		\item \begin{tabular}{|c|c|c|c|}
			\hline
			1 & 2 & 3 & 4 \\\hline
			3 & 4 & 1 & 2 \\\hline
			2 & 1 & 4 & 3 \\\hline
			4 & 3 & 2 & 1 \\\hline
		\end{tabular}
		\item 
		\leavevmode
		\begin{enumerate}
			\item In CNF, we get the following expression:
			\begin{multline*}
			(\neg S143 \lor \neg S243)\land (\neg S143 \lor \neg S343)\land (\neg S143 \lor \neg S443)\land (\neg S243 \lor \neg S343)\land \\ (\neg S243 \lor \neg S443)\land (\neg S343 \lor \neg S443)\land (S143 \lor S243 \lor S343 \lor S443) 
			\end{multline*}
			All the expressions of the form $(\neg Sx43 \lor \neg Sy43)$ are saying that a three cannot be in two squares at the same time. The expression
			\[ (S143 \lor S243 \lor S343 \lor S443) \]
			is saying that the three has to be in one of the four squares. 
			\item To prove that a 3 has to be in the third row of the fourth column, we will use the following premises:
			\begin{multline*}
			(\neg S143 \lor \neg S243)\land (\neg S143 \lor \neg S343)\land (\neg S143 \lor \neg S443)\land (\neg S243 \lor \neg S343)\land \\ (\neg S243 \lor \neg S443)\land (\neg S343 \lor \neg S443)\land (S143 \lor S243 \lor S343 \lor S443) 
			\end{multline*}
			and
			\begin{equation*}
			(S144)\land(S242)\land(S441)\land(S111)\land(S312)\land(S423)\land(\neg S143)\land(\neg S243)\land(\neg S443)
			\end{equation*}
			This second premise is just the initial placement of numbers along with the fact that a 3 cannot lie in any of the squares that already have numbers in them in the fourth row. Finally, for our contraction, we will assume the premise:
			\begin{equation*}
			\neg S343.
			\end{equation*}
			Then, using the following resolutions, we derive a contradiction:
			\begin{align*}
			&\frac{(S143 \lor S243 \lor S343 \lor S443),(\neg S343)}{(S143 \lor S243 \lor S443)} \\
			&\frac{(S143 \lor S243 \lor S443),(\neg S143)}{(S243 \lor S443)} \\
			&\frac{(S243 \lor S443),(\neg S243)}{(S443)} \\
			&\frac{(S443),(\neg S443)}{\{\}}
			\end{align*}
		\end{enumerate}
		Thus, there must be a 3 in the third row of the fourth column.
	\end{enumerate}
	\setcounter{section}{3}
	\section{(30 Points) Mastermind}
	\lstinputlisting[language=Python]{masterWriteup.py}
	Again, the code above is only the important parts so it will not run as is. The main part strategy this program uses is removing potential guess that cannot possibly be right based on the previous guess and response. Each time the computer receives a response to a guess, it loops through the remaining guesses to determine which ones to keep. For each guess remaining, we see if the previous guess would have gotten the same response if that guess were the code. If it wouldn't have gotten the same response, then we can remove this guess from our remaining guess list. The logic behind this is as follows. One of the remaining guesses is the correct code and it gave a certain response $x$ to the previous guess. Then, only the set of guesses that would also give the same response could possibly be correct codes. This drastically cuts down on the numbers of guesses to make and is what allows this problem to become tractable in a small number of moves. Once all the incorrect guesses have been removed, the game just chooses the first one in the remaining list as its next guess. This strategy works well for the small game size since the number of states decreases quickly to a manageable number. 
	
	This strategy would not work for a game with 30 colors and 25 positions. The number of possible codes is 
	\[30^{25} = 8.47 \cdot 10^{36}.\]
	If you could store each code in a single byte, then it would take $8.47 \cdot 10^12$ yottabytes to store all of the possible codes. This is entirely unfeasible with current hardware. 
\end{document}